\documentclass[a4paper,14pt]{extarticle}
\usepackage{graphicx}
\usepackage{ucs}
\usepackage[utf8x]{inputenc}
\usepackage[russian]{babel}
\usepackage{multirow}
\usepackage{mathtext}
\usepackage[T2A]{fontenc}
\usepackage{titlesec}
\usepackage{float}
\usepackage{empheq}
\usepackage{amsfonts}
\usepackage{amsmath}
\title{\textbf{Лабораторная работа по взятию производной }}\author{Чурсин Владимир Б01-305}
\begin{document}
\maketitle
\section{\ Функция \\}$\frac{\ln{(((x^2)+x))}}{(x+2)}$ \newline \\ 
Заметим, что: \\ 

$(x+2)' = 1+0$ \newline \\ 
Необходимые условия выпуклости: \\ 

$(x^2)' = (2*(x^1))*1$ \newline \\ 
Необходимо сделать предостережение о неверном применении правила Лопиталя: \\ 

$((x^2)+x)' = ((2*(x^1))*1)+1$ \newline \\ 
Очевидно, что: \\ 

$(\ln{(((x^2)+x))})' = \frac{1}{((x^2)+x)}*(((2*(x^1))*1)+1)$ \newline \\ 
Необходимые условия выпуклости: \\ 

$(\frac{\ln{(((x^2)+x))}}{(x+2)})' = \frac{(((\frac{1}{((x^2)+x)}*(((2*(x^1))*1)+1))*(x+2))-(\ln{(((x^2)+x))}*(1+0)))}{((x+2)^2)}$ \newline \\ 
Итого имеем: \\
$(\frac{\ln{(((x^2)+x))}}{(x+2)})' = \frac{(((\frac{1}{((x^2)+x)}*((2*x)+1))*(x+2))-\ln{(((x^2)+x))})}{((x+2)^2)}$ \newline \\ 


\end{document}